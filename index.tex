% Options for packages loaded elsewhere
% Options for packages loaded elsewhere
\PassOptionsToPackage{unicode}{hyperref}
\PassOptionsToPackage{hyphens}{url}
\PassOptionsToPackage{dvipsnames,svgnames,x11names}{xcolor}
%
\documentclass[
  letterpaper,
  DIV=11,
  numbers=noendperiod]{scrartcl}
\usepackage{xcolor}
\usepackage{amsmath,amssymb}
\setcounter{secnumdepth}{5}
\usepackage{iftex}
\ifPDFTeX
  \usepackage[T1]{fontenc}
  \usepackage[utf8]{inputenc}
  \usepackage{textcomp} % provide euro and other symbols
\else % if luatex or xetex
  \usepackage{unicode-math} % this also loads fontspec
  \defaultfontfeatures{Scale=MatchLowercase}
  \defaultfontfeatures[\rmfamily]{Ligatures=TeX,Scale=1}
\fi
\usepackage{lmodern}
\ifPDFTeX\else
  % xetex/luatex font selection
\fi
% Use upquote if available, for straight quotes in verbatim environments
\IfFileExists{upquote.sty}{\usepackage{upquote}}{}
\IfFileExists{microtype.sty}{% use microtype if available
  \usepackage[]{microtype}
  \UseMicrotypeSet[protrusion]{basicmath} % disable protrusion for tt fonts
}{}
\makeatletter
\@ifundefined{KOMAClassName}{% if non-KOMA class
  \IfFileExists{parskip.sty}{%
    \usepackage{parskip}
  }{% else
    \setlength{\parindent}{0pt}
    \setlength{\parskip}{6pt plus 2pt minus 1pt}}
}{% if KOMA class
  \KOMAoptions{parskip=half}}
\makeatother
% Make \paragraph and \subparagraph free-standing
\makeatletter
\ifx\paragraph\undefined\else
  \let\oldparagraph\paragraph
  \renewcommand{\paragraph}{
    \@ifstar
      \xxxParagraphStar
      \xxxParagraphNoStar
  }
  \newcommand{\xxxParagraphStar}[1]{\oldparagraph*{#1}\mbox{}}
  \newcommand{\xxxParagraphNoStar}[1]{\oldparagraph{#1}\mbox{}}
\fi
\ifx\subparagraph\undefined\else
  \let\oldsubparagraph\subparagraph
  \renewcommand{\subparagraph}{
    \@ifstar
      \xxxSubParagraphStar
      \xxxSubParagraphNoStar
  }
  \newcommand{\xxxSubParagraphStar}[1]{\oldsubparagraph*{#1}\mbox{}}
  \newcommand{\xxxSubParagraphNoStar}[1]{\oldsubparagraph{#1}\mbox{}}
\fi
\makeatother


\usepackage{longtable,booktabs,array}
\usepackage{calc} % for calculating minipage widths
% Correct order of tables after \paragraph or \subparagraph
\usepackage{etoolbox}
\makeatletter
\patchcmd\longtable{\par}{\if@noskipsec\mbox{}\fi\par}{}{}
\makeatother
% Allow footnotes in longtable head/foot
\IfFileExists{footnotehyper.sty}{\usepackage{footnotehyper}}{\usepackage{footnote}}
\makesavenoteenv{longtable}
\usepackage{graphicx}
\makeatletter
\newsavebox\pandoc@box
\newcommand*\pandocbounded[1]{% scales image to fit in text height/width
  \sbox\pandoc@box{#1}%
  \Gscale@div\@tempa{\textheight}{\dimexpr\ht\pandoc@box+\dp\pandoc@box\relax}%
  \Gscale@div\@tempb{\linewidth}{\wd\pandoc@box}%
  \ifdim\@tempb\p@<\@tempa\p@\let\@tempa\@tempb\fi% select the smaller of both
  \ifdim\@tempa\p@<\p@\scalebox{\@tempa}{\usebox\pandoc@box}%
  \else\usebox{\pandoc@box}%
  \fi%
}
% Set default figure placement to htbp
\def\fps@figure{htbp}
\makeatother





\setlength{\emergencystretch}{3em} % prevent overfull lines

\providecommand{\tightlist}{%
  \setlength{\itemsep}{0pt}\setlength{\parskip}{0pt}}



 


\KOMAoption{captions}{tableheading}
\makeatletter
\@ifpackageloaded{bookmark}{}{\usepackage{bookmark}}
\makeatother
\makeatletter
\@ifpackageloaded{caption}{}{\usepackage{caption}}
\AtBeginDocument{%
\ifdefined\contentsname
  \renewcommand*\contentsname{Table of contents}
\else
  \newcommand\contentsname{Table of contents}
\fi
\ifdefined\listfigurename
  \renewcommand*\listfigurename{List of Figures}
\else
  \newcommand\listfigurename{List of Figures}
\fi
\ifdefined\listtablename
  \renewcommand*\listtablename{List of Tables}
\else
  \newcommand\listtablename{List of Tables}
\fi
\ifdefined\figurename
  \renewcommand*\figurename{Figure}
\else
  \newcommand\figurename{Figure}
\fi
\ifdefined\tablename
  \renewcommand*\tablename{Table}
\else
  \newcommand\tablename{Table}
\fi
}
\@ifpackageloaded{float}{}{\usepackage{float}}
\floatstyle{ruled}
\@ifundefined{c@chapter}{\newfloat{codelisting}{h}{lop}}{\newfloat{codelisting}{h}{lop}[chapter]}
\floatname{codelisting}{Listing}
\newcommand*\listoflistings{\listof{codelisting}{List of Listings}}
\makeatother
\makeatletter
\makeatother
\makeatletter
\@ifpackageloaded{caption}{}{\usepackage{caption}}
\@ifpackageloaded{subcaption}{}{\usepackage{subcaption}}
\makeatother
\usepackage{bookmark}
\IfFileExists{xurl.sty}{\usepackage{xurl}}{} % add URL line breaks if available
\urlstyle{same}
\hypersetup{
  pdftitle={Imarika Discipleship Class - Scripture Memory and Meditation},
  pdfauthor={B. Kamara},
  colorlinks=true,
  linkcolor={blue},
  filecolor={Maroon},
  citecolor={Blue},
  urlcolor={Blue},
  pdfcreator={LaTeX via pandoc}}


\title{Imarika Discipleship Class - Scripture Memory and Meditation}
\author{B. Kamara}
\date{Invalid Date}
\begin{document}
\maketitle

\renewcommand*\contentsname{Table of contents}
{
\hypersetup{linkcolor=}
\setcounter{tocdepth}{2}
\tableofcontents
}

\bookmarksetup{startatroot}

\chapter{Imarika Discipleship: Scripture Memory and
Meditation}\label{imarika-discipleship-scripture-memory-and-meditation}

\section{Preface}\label{preface}

These notes are from the second session of the \emph{Imarika
Discipleship Class}, a series designed to ground believers in
foundational spiritual practices. ``Imarika'' (Swahili for ``be firm''
or ``stand strong'') reflects the goal of this class: to strengthen
faith through intentional engagement with God's Word.

In this session, we focus on \textbf{Scripture Memory and Meditation},
exploring: - Why Scripture Memory - Is it necessary? - How to memorise
scripture? - Meditation - How to Meditate on scripture.

These teachings, delivered by \emph{Brian Kamara} on \emph{12th April
2025}, a call to action---to internalize and live out the truths of
Scripture.

\begin{center}\rule{0.5\linewidth}{0.5pt}\end{center}

\bookmarksetup{startatroot}

\chapter{Scripture Memory and
Meditation}\label{scripture-memory-and-meditation}

\section{Why Scripture Memory.}\label{why-scripture-memory.}

Ephesians 6:17. The Sword of the Spirit. We are in a war, and the
Scripture we have in our heart is the ammunition.

Imagine facing a decision and needing to find guidance. When the Holy
Spirit looks for weapons, all He finds is John 3:16 and a few of the
commandments memorized in the wrong order.

\subsection{Is it necessary?}\label{is-it-necessary}

Psalms 119:9-16; 23-24 - Benefits of meditating on Scripture

1 Peter 3:15 - Witnessing.

How did it help the men of the bible?

\begin{enumerate}
\def\labelenumi{\arabic{enumi}.}
\tightlist
\item
  Jesus - Matthew 4:1-11 - Jesus speaking with the evil one\\
\item
  Peter - Acts 2:14-40 - Peter preaching after the Pentecost\\
\item
  Moses - Numbers 14:17-20
\end{enumerate}

\subsection{How to memorise scripture?}\label{how-to-memorise-scripture}

\begin{enumerate}
\def\labelenumi{\arabic{enumi}.}
\item
  Have a plan.\\
\item
  Speak aloud and write out the verse\\
\item
  Memorise the verse word perfect.\\
\item
  Avoid excuses.

  \begin{enumerate}
  \def\labelenumii{\alph{enumii}.}
  \tightlist
  \item
    I have bad memory. What if I offered you 1000 Ksh. for every word.\\
  \item
    I am too busy. A man was able to memorize while on the job.\\
  \item
    If I memorize too many, I may become proud.
  \end{enumerate}
\item
  Be disciplined. Better is a man who rules his spirit than one who
  takes over a nation.
\item
  Find an accountability partner.
\item
  Revise and meditate daily. What is measured, grows.
\end{enumerate}

However, scripture memory alone is not what will save. The Pharisees
also memorized scripture and quoted it to Jesus. Satan also memorized
scripture.

\bookmarksetup{startatroot}

\chapter{Meditation}\label{meditation}

Meditation is deep unrushed thinking about the truths revealed in
Scripture in order to \textbf{understand, believe, and apply} them in
our lives.

Joshua 1:8\\
Psalms 1:1-3\\
Psalms 119:97-100

Oh, how I love your instructions! I think about them all day long. Your
commands make me wiser than my enemies, for they are my constant guide.
Yes, I have more insight than my teachers, for I am always thinking of
your laws.I am even wiser than my elders, for I have kept your
commandments.

Rewriting a verse in your own words can really help you put into words
the thoughts you have.

\section{How to Meditate on
scripture.}\label{how-to-meditate-on-scripture.}

\begin{enumerate}
\def\labelenumi{\arabic{enumi}.}
\tightlist
\item
  Select and rewrite the verse that you are studying.
\item
  Write out the message of the verse.
\item
  Look at the context of the verse
\item
  Write down questions that arise from the verse.
\item
  Application. Ask God to show you how to apply the verse to your life.
\item
  Proof of memory. Try and memorize the verse.
\item
  Sharing with people. Whenever you speak with someone, encourage them
  with what you are learning.
\end{enumerate}

\bookmarksetup{startatroot}

\chapter{Group Discussion Questions.}\label{group-discussion-questions.}

\begin{enumerate}
\def\labelenumi{\arabic{enumi}.}
\item
  Different people memorize Bible verses for different reasons. Why do
  you\\
  want to begin/continue memorizing Scripture?
\item
  If I continue to memorize one verse of Scripture every week of the
  year, what\\
  benefits would I experience?
\item
  If I were to make this a lifelong habit, what benefits would I enjoy?
\item
  Has memorizing scripture ever helped you in your life? Kindly share
  with us.
\item
  Let's think together. How will I commit to memorize scripture? How can
  we\\
  keep each other accountable?
\end{enumerate}

Read this passage of scripture together.\\
Psalms 19:7-11




\end{document}
